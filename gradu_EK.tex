\documentclass[utf8]{gradu3}
% Jos työ on kandidaatintutkielma eikä pro gradu, käytä ylläolevan asemesta
%\documentclass[utf8,bachelor]{gradu3}
% Jos kirjoitat englanniksi, käytä ylläolevan asemesta
%\documentclass[utf8,english]{gradu3}
% tai
%\documentclass[utf8,bachelor,english]{gradu3}

\usepackage{graphicx} % kuvien mukaan ottamista varten

\usepackage{amsmath} % hyödyllinen jos tekstisi sisältää matikkaa,
                     % ei pakollinen

\usepackage{booktabs} % hyvä kauniiden taulukoiden tekemiseen

\usepackage[authordate,backend=biber,noibid]{biblatex-chicago} % biber / chicago-tyylin käyttö

% HUOM! Tämän tulee olla viimeinen \usepackage koko dokumentissa!
\usepackage[bookmarksopen,bookmarksnumbered,linktocpage]{hyperref}

\addbibresource{gradu_EK.bib} % Lähdetietokannan tiedostonimi

\begin{document}

\title{Vertaileva tutkimus koneoppimisen hyödyntämisestä videopelien reitinhaussa}
\translatedtitle{Comparative study of utilizing machine learning in video games' pathfinding}
\studyline{Tietotekniikka}
\tiivistelma{%
TODO: tiivistelmä suomeksi
}
\abstract{%
TODO: In english
}

\author{Emil Keränen}
\contactinformation{\texttt{emil.a.keranen@student.jyu.fi}}
% jos useita tekijöitä, anna useampi \author-komento
\supervisor{Tommi Kärkkäinen}
% jos useita ohjaajia, anna useampi \supervisor-komento
\avainsanat{koneoppiminen, videopeli, reitinhaku, syvä vahvistusoppiminen}
\keywords{machine learning, video game, pathfinding, deep reinforcement learning, Soft Actor Critic, Machine Learning Agents, Unity}

\maketitle

\bigskip

\mainmatter

\chapter{Johdanto}

- Tutkimuksen kohteena koneoppimisen tehostama reitinhaku videopeleissä
 ja sen vertaaminen heuristiseen A*-algoritmiin.

- Yleisesti reitinhaulla tarkoitetaan alku- ja loppupisteen välisen reitin selvittämistä. Useimmiten tarkoituksena
on löytää lyhin reitti väistellen samalla matkan varrella olevia esteitä.

- Reitinhakua tarvitaan videopelien lisäksi myös mm. robotiikassa. 

- Videopeleissä reitinhaku ilmenee pääasiassa tekoälyagenttien suorittamana toimintana, joten tässä tutkimuksessa keskitytään agentteihin.
Näitä agentteja kutsutaan myös ei-pelaaja-hahmoiksi (engl. non-player-character, NPC).

- Ei-pelaaja-hahmot ja itseasiassa videopelien reitinhaku vertautuvat hyvin robotiikkaan ja robottien reitinhakuun.

- Sekä robotiikassa että videopeleissä toiminta-alue voi muuttua hyvinkin paljon reaaliajassa, jolloin reitinhaun täytyy
sopeutua muutoksiin nopeasti. Käytetyt ratkaisut sen sijaan voivat vaihdella näiden kahden osa-alueen välillä: robotiikassa
tarkkuus ja turvallisuus nousevat tärkeimmiksi ominaisuuksiksi ja vastaavasti videopeleissä nopeus määrittää reitinhaun "hyvyyden".

- Reitinhaku on aina ollut vaativa ongelma videopeleissä, mutta nykyään reitinhaun ongelmallisuus voidaan
useimmissa tapauksissa sivuuttaa laatimalla heuristinen ratkaisu A*-algoritmin avulla.

- Koneoppiminen mahdollistaa aiemman kokemuksen hyödyntämisen myöhemmässä toiminnassa.
Agentteja voidaan kouluttaa harjoitteludatan avulla, jolloin ne oppivat toimimaan tuntemattomissa tilanteissa.
Koneoppimisen ansiosta reitinhaku-agentti voidaan opettaa toimimaan vaativissa ja
dynaamisissa pelialueissa, joissa muuttuvat esteet ja alueen labyrinttimäisyys
heikentävät A*-algoritmin toimintaa.

- Tutkimuksen ideana on käyttää Unity-pelimoottorille luotuja koneoppimisagentteja (engl. Unity Machine Learning Agents)
ja opettaa niitä erilaisten pelialueiden avulla. Opettamisen jälkeen agentteja testataan oikeilla
pelialueilla ja verrataan tuloksia A*-algoritmilla saatuihin tuloksiin.

- ML-agents perustuu PyTorch-kirjastoon ja mahdollistaa vahvistusoppimisen hyödyntämisen.

- Tensorboard-lisäosan avulla voidaan visualisoida palkkioiden keskiarvot ja opetuksen edistyminen opetuksen aikana.

- Pelialueiden on tarkoitus olla monimutkaisia ja dynaamisia, koska A*-algoritmi suoriutuu
yksinkertaisista reitinhakutehtävistä moitteettomasti.

\chapter{Reitinhaku videopeleissä}

Reitinhaku on yksi videopelien tekoälyn tunnetuimmista ja haastavimmista ongelmista, jota on tutkittu jo vuosikymmenten ajan. Entistä tehokkaammat laitteet ja laskennallisesti vaativammat reitinhakuongelmat ovat tuoneet monia eri ratkaisuja, joista tunnetuin on laajasti käytetty A*-algoritmi ja sen variaatiot, kuten Theta*- ja Phi*-algoritmit.

\chapter{Unity}

Unity on Unity Technologiesin kehittämä pelinkehitysalusta, joka sisältää oman renderöinti- ja fysiikkamoottorin sekä Unity Editor -nimisen graafisen käyttöliittymän \parencite{juliani2018unity}. Unityllä on mahdollista kehittää perinteisten 3D- ja 2D-pelien lisäksi myös esimerkiksi VR-pelejä tietokoneille, mobiililaitteille ja pelikonsoleille. Unitystä onkin vuosien mittaan tullut yksi tunnetuimmista pelinkehitysalustoista, jonka parissa työskentelee kuukausittain jopa 1.5 miljoonaa aktiivista käyttäjää \parencite{unityweb}.

Viime vuosina Unityä on käytetty simulointialustana tekoälytutkimuksen parissa. Unity mahdollistaa lähes mielivaltaisten tilanteiden ja ympäristöjen simuloinnin 2D ruudukkokartoista monimutkaisiin pulmanratkaisutehtäviin, joka on sen suurimpia vahvuuksia simulointialustana. Kehitystyö ja prototypointi ovat Unityllä myös erityisen nopeaa. \parencite{juliani2018unity}.

\section{Machine Learning Agents}

Machine Learning Agents on Unitylle kehitetty koneoppimispaketti, jonka avulla peliin voidaan ottaa käyttöön koneoppimisagentteja.

\chapter{Koneoppiminen}

Soft Actor-Critic on Haarnojan ym. kehittämä syvä vahvistusoppimis algoritmi \parencite{haarnoja2018soft}. 

\chapter{Tutkimusstrategia/metodi ja sen valintaperusteet}

- Empiirinen vertaileva tutkimus?

- Luodaan pelialueita, toteutetaan heuristinen A*-algoritmi, opetetaan koneoppimisagentit syvän vahvistetun oppimisen avulla (Soft Actor Critic -algoritmi)
ja sijoitetaan koneoppimisagentit pelialueille. Tämän jälkeen ratkaistaan reitinhakutehtävä erikseen molemmilla menetelmillä ja verrataan saatuja tuloksia keskenään
(mm. lasketaan tehtävään kulunut aika, suoriutuiko tehtävästä tietyssä ajassa vai ei).

\chapter{Aineiston keruun suunnittelu}

- Agentit koneopetetaan käyttäen Unity ML-agents -pakettia. Opetusprosessista saatu malli-tiedosto (model, .onnx-tiedosto) liitetään agentin
komponentiksi, jolloin agentti voi toimia pelialueella itsenäisesti.

- Tensorboardin avulla oppimisprosessia voidaan visualisoida esimerkiksi palkkioiden suhteen.

- Aineisto kerätään peliagenteilta jokaisen reitinhakuongelman aikana. Vähintään aika ja tieto siitä onnistuiko agentti tai A*-algoritmi otetaan talteen.
Myös Tensorboardin muodostamia kuvaajia voidaan käyttää apuna koneoppimista arvioidessa (mm. palkkioiden keskiarvo opetuksessa).

- Ajan laskemiseen käytetään Unityn valmiita kirjastoja.

\chapter{Aineiston keruu}

- Reitinhakuongelmia on jokaista erilaista pelialuetta kohden yksi. Pelialueita luodaan ainakin muutama kymmen.

- Malli-tiedostoja (model, .onnx-tiedosto) luodaan muutamia eri parametreilla, jolloin voidaan verrata parametrien vaikutusta agentin suoriutumiseen.

- Ajan laskeminen alkaa, kun agentti käsketään siirtymään pisteestä A pisteeseen B.
Ajan laskeminen loppuu, kun agentti saapuu pisteeseen B. Jos agentti ei syystä tai toisesta
pysty ratkaisemaan ongelmaa eli ei saavuta pistettä B (esim. aikamaksimi saavutetaan, alueesta riippuen 20+ sekuntia), merkitään ratkaisun tilaksi "epätosi"
ja jätetään aika-arvo tyhjäksi. Jos ratkaisu löytyy, merkitään ratkaisun tilaksi "tosi" ja
asetetaan aika-arvoksi mitattu aika. Sama prosessi toistetaan A*-algoritmilla.

- Jos koneoppimisagentteja opetetaan eri määrällä dataa, ilmoitetaan myös opetusdatan määrä.

- Kaikki data kerätään yhteen tiedostoon ja käsitellään jälkikäteen.

\chapter{Aineiston analyysi}

- Kuvaajien avulla visualisoidaan kuluneita aikoja ja vertaillaan saatujen aikojen erotuksia.

- TODO: lisätietoja analyysista

\chapter{Tulokset}

- TODO

\chapter{Johtopäätökset}

- TODO

\printbibliography

- Panov, A., Yakovlev, K. ja Suvorov, R., Grid Path Planning with Deep
Reinforcement Learning: Preliminary Results, Procedia Computer Science
Volume 123, Pages 347-353, 2018,
https://doi.org/10.1016/j.procs.2018.01.054

- X. Lei, Z. Zhang ja P. Dong, Dynamic Path Planning of Unknown
Environment Based on Deep Reinforcement Learning, 2018,
https://doi.org/10.1155/2018/5781591

- TODO: oikea merkintätapa ja muita lähteitä.

\chapter{Liitteet}

- Kuvat tai mallinnokset pelialueesta.

- Tensorboardin kuvaajat mm. agentin palkkioiden kehityksestä.

\end{document}
